%!TEX root = main_acm.tex

\section{Related Work}
\label{sec:rel}
Anycast deployment and performance have been characterized and evaluated by different active probing methods. Madory \emph{et al.}~\cite{madory2013anycasters} use geolocation of transit IP and geo-inconsistency to detect anycast prefixes. Cicalese \emph{et al.} ~\cite{cicalese2015characterizing, cicalese2015fistful, cicalese2018longitudinal} propose a method for enumeration and geolocation of anycast instances based on latency measurements. Vries \emph{et al.}~\cite{deVries:2017} propose a method that maps anycast catchments via active probes to provide better coverage. 

\textbf{Anycast-based Internet Services}.
Fan \emph{et al.} ~\cite{fan2013} combine the CHAOS queries with traceroutes and use new IN records to support open recursive DNS servers as vantage points to detect and study anycast-based DNS infrastructures.
Calder \emph{et al.} ~\cite{Calder:2015} study the performance of an anycast CDN and find that some clients are directed to a sub-optimal front-end.
Moura \emph{et al.} ~\cite{moura2016anycast} study the Nov. 2015 event of Root DNS attacked by DDoS from the anycast's perspective. 
Giordano \emph{et al.}~\cite{giordano2016first} perform a passive characterization study on anycast traffic in CDNs and present temporal properties, service diversity, and deployments of anycast traffic.

Schmidt \emph{et al.} ~\cite{de2017anycast} investigate the relationship between IP anycast and latency from four Root DNS nameservers. Their key results show that geographic location and connectivity have a stronger impact on latency than the number of sites. 
Li \emph{et al.}~\cite{li2018internet} perform a study on anycast's route selection and performance using D-root Server traces, and they validate that equal-length AS paths are the main reason for anycast latency inflation.
Wei \emph{et al.}~\cite{wei2018does} study the service (in)stability of anycast services. They confirm that a small number of users are affected by the instability of anycast, potentially caused by the load balancers on the path.

\textbf{Remote Peering}.
Castro \emph{et al.}~\cite{castro2014remote} present a systematic study of remote peering at IXPs using ping-based methods. They discuss the impact of remote peering on Internet reliability, security, and economies.
Nomikos \emph{et al.}~\cite{Nomikos18} perform a comprehensive measurement study
of remote peering, and they achieve very high accuracy and coverage levels by
combining RTT measurements with other domain-specific information like facility
locations, IXP port capacity, and private connectivity. They study the features
and trends of remote peering, showing that remote peering may route traffic to
more distant destinations. Their work does not focus on anycast prefixes though.

